\batchmode


\documentclass[12pt,leqno,a4paper,spanish]{book}
\RequirePackage{ifthen}


\usepackage[dvips]{graphicx}
\usepackage[dvipsnames,usenames]{color}
\usepackage{makeidx}
\usepackage[absolute]{textpos}
\usepackage{wrapfig}
\usepackage{eso-pic}
\usepackage{xcolor}
\usepackage[no-math]{fontspec}
\usepackage{mathabx}
\usepackage{polyglossia}
\setmainlanguage{spanish}
\usepackage{listings}
\usepackage[colorlinks,unicode]{hyperref}
\usepackage{soul}
\usepackage{xeCJK}
\usepackage{placeins}
\usepackage{pbox}
\usepackage{tikz}
\usepackage{pgfornament}
\usetikzlibrary{trees}
\usepackage{html}
\hypersetup{
    pdfauthor={Jos\'{e} Miguel Gonz\'{a}lez Aguilera},
    pdftitle={Inmersi\'{o}n en Python 3},
    pdfsubject={Programaci\'{o}n en Python 3, versión 1 en español},
    pdfkeywords={python,python 3,programaci\'{o}n}
}
\parindent 1cm
\parskip 0.2cm
\topmargin 0.2cm
\oddsidemargin 1cm
\evensidemargin 0.5cm
\textwidth 15cm
\textheight 21cm


\graphicspath{{./imagen/}}



\hyphenation{Python} 

%
\providecommand{\cajaTexto}[1]{\begin{wrapfigure}{r}{.4\linewidth}\fbox{\colorbox{gray}{\parbox{.9\linewidth} {#1}}}\end{wrapfigure}} 

%
\providecommand{\cajaTextoAncho}[1]{\fbox{\colorbox{gray}{\parbox{0.9\linewidth} {#1}}}} 



%
\newenvironment{citaCap}{\begin{flushright}\begin{itshape}}
{\end{itshape}\end{flushright}} 



%
\newenvironment{listing}{\begin{list}{}{\setlength{\leftmargin}{1em}}\item\footnotesize\samepage}
{\end{list}} 

%
\providecommand{\ac}[1]{\textrm{\'{#1}}}%
\providecommand{\til}[1]{\textrm{\~{#1}}} 

%
\providecommand{\codigo}[1]{\textsf{#1}} 

%
\providecommand{\difl}{$\blackdiamond\diamond\diamond\diamond\diamond$} 

%
\providecommand{\difll}{$\blackdiamond\blackdiamond\diamond\diamond\diamond$} 

%
\providecommand{\diflll}{$\blackdiamond\blackdiamond\blackdiamond\diamond\diamond$} 

%
\providecommand{\difllll}{$\blackdiamond\blackdiamond\blackdiamond\blackdiamond\diamond$} 

%
\providecommand{\diflllll}{$\blackdiamond\blackdiamond\blackdiamond\blackdiamond\blackdiamond$} 


\renewenvironment{lstlisting}%
{\rawhtml <pre>\endrawhtml\verbatim}%
{\endverbatim\rawhtml </pre>\endrawhtml}


%
\renewcommand{\difl}{1 sobre 5}
      %
\renewcommand{\difll}{2 sobre 5}
      %
\renewcommand{\diflll}{3 sobre 5}
      %
\renewcommand{\difllll}{4 sobre 5}
      %
\renewcommand{\diflllll}{5 sobre 5}
      %
\renewcommand{\cajaTexto}[1]{\rawhtml <blockquote>#1</blockquote> \endrawhtml}
      %
\renewcommand{\cajaTextoAncho}[1]{\rawhtml <blockquote>#1</blockquote> \endrawhtml}
      %
\renewcommand{\sqrt}[1]{sqrt(#1)}
\setcounter{chapter}{-2}

%
\providecommand{\code}{\textcolor{OliveGreen}\bfseries} 


\title{Inmersión en Python 3}
\author{Mark Pilgrim}


\makeindex


\definecolor{gray}{rgb}{0.98,0.98,0.98}
\definecolor{black}{rgb}{0,0,0}


\widowpenalty=10000
\clubpenalty=10000
\raggedbottom


\lstset{language=Python,showstringspaces=false,numbers=left,
        numberstyle=\footnotesize ,backgroundcolor=\color{gray},
        rulesep=1pt, rulesepcolor=\color{black},frame=leftline,
        basicstyle=\footnotesize , mathescape=true}





\pagecolor[gray]{.7}

\usepackage[latin1]{inputenc}



\makeatletter
\AtBeginDocument{\makeatletter
\input /home/jmgaguilera/Documentos/inmersionenpython3/intermedio/inmersionEnPython.aux
\makeatother
}
\AtBeginDocument{\makeatletter
\input /home/jmgaguilera/Documentos/inmersionenpython3/intermedio/portada.aux
\makeatother
}
\AtBeginDocument{\makeatletter
\input /home/jmgaguilera/Documentos/inmersionenpython3/intermedio/cap-1.aux
\makeatother
}
\AtBeginDocument{\makeatletter
\input /home/jmgaguilera/Documentos/inmersionenpython3/intermedio/cap0.aux
\makeatother
}
\AtBeginDocument{\makeatletter
\input /home/jmgaguilera/Documentos/inmersionenpython3/intermedio/cap1.aux
\makeatother
}
\AtBeginDocument{\makeatletter
\input /home/jmgaguilera/Documentos/inmersionenpython3/intermedio/cap2.aux
\makeatother
}
\AtBeginDocument{\makeatletter
\input /home/jmgaguilera/Documentos/inmersionenpython3/intermedio/cap3.aux
\makeatother
}
\AtBeginDocument{\makeatletter
\input /home/jmgaguilera/Documentos/inmersionenpython3/intermedio/cap4.aux
\makeatother
}
\AtBeginDocument{\makeatletter
\input /home/jmgaguilera/Documentos/inmersionenpython3/intermedio/cap5.aux
\makeatother
}
\AtBeginDocument{\makeatletter
\input /home/jmgaguilera/Documentos/inmersionenpython3/intermedio/cap6.aux
\makeatother
}
\AtBeginDocument{\makeatletter
\input /home/jmgaguilera/Documentos/inmersionenpython3/intermedio/cap7.aux
\makeatother
}
\AtBeginDocument{\makeatletter
\input /home/jmgaguilera/Documentos/inmersionenpython3/intermedio/cap8.aux
\makeatother
}
\AtBeginDocument{\makeatletter
\input /home/jmgaguilera/Documentos/inmersionenpython3/intermedio/cap9.aux
\makeatother
}
\AtBeginDocument{\makeatletter
\input /home/jmgaguilera/Documentos/inmersionenpython3/intermedio/cap10.aux
\makeatother
}
\AtBeginDocument{\makeatletter
\input /home/jmgaguilera/Documentos/inmersionenpython3/intermedio/cap11.aux
\makeatother
}
\AtBeginDocument{\makeatletter
\input /home/jmgaguilera/Documentos/inmersionenpython3/intermedio/cap12.aux
\makeatother
}
\AtBeginDocument{\makeatletter
\input /home/jmgaguilera/Documentos/inmersionenpython3/intermedio/cap13.aux
\makeatother
}
\AtBeginDocument{\makeatletter
\input /home/jmgaguilera/Documentos/inmersionenpython3/intermedio/cap14.aux
\makeatother
}
\AtBeginDocument{\makeatletter
\input /home/jmgaguilera/Documentos/inmersionenpython3/intermedio/cap15.aux
\makeatother
}
\AtBeginDocument{\makeatletter
\input /home/jmgaguilera/Documentos/inmersionenpython3/intermedio/cap16.aux
\makeatother
}
\AtBeginDocument{\makeatletter
\input /home/jmgaguilera/Documentos/inmersionenpython3/intermedio/apendiceA.aux
\makeatother
}
\AtBeginDocument{\makeatletter
\input /home/jmgaguilera/Documentos/inmersionenpython3/intermedio/apendiceB.aux
\makeatother
}
\AtBeginDocument{\makeatletter
\input /home/jmgaguilera/Documentos/inmersionenpython3/intermedio/apendiceC.aux
\makeatother
}
\AtBeginDocument{\makeatletter
\input /home/jmgaguilera/Documentos/inmersionenpython3/intermedio/apendiceD.aux
\makeatother
}

\makeatletter
\count@=\the\catcode`\_ \catcode`\_=8 
\newenvironment{tex2html_wrap}{}{}%
\catcode`\<=12\catcode`\_=\count@
\newcommand{\providedcommand}[1]{\expandafter\providecommand\csname #1\endcsname}%
\newcommand{\renewedcommand}[1]{\expandafter\providecommand\csname #1\endcsname{}%
  \expandafter\renewcommand\csname #1\endcsname}%
\newcommand{\newedenvironment}[1]{\newenvironment{#1}{}{}\renewenvironment{#1}}%
\let\newedcommand\renewedcommand
\let\renewedenvironment\newedenvironment
\makeatother
\let\mathon=$
\let\mathoff=$
\ifx\AtBeginDocument\undefined \newcommand{\AtBeginDocument}[1]{}\fi
\newbox\sizebox
\setlength{\hoffset}{0pt}\setlength{\voffset}{0pt}
\addtolength{\textheight}{\footskip}\setlength{\footskip}{0pt}
\addtolength{\textheight}{\topmargin}\setlength{\topmargin}{0pt}
\addtolength{\textheight}{\headheight}\setlength{\headheight}{0pt}
\addtolength{\textheight}{\headsep}\setlength{\headsep}{0pt}
\setlength{\textwidth}{349pt}
\newwrite\lthtmlwrite
\makeatletter
\let\realnormalsize=\normalsize
\global\topskip=2sp
\def\preveqno{}\let\real@float=\@float \let\realend@float=\end@float
\def\@float{\let\@savefreelist\@freelist\real@float}
\def\liih@math{\ifmmode$\else\bad@math\fi}
\def\end@float{\realend@float\global\let\@freelist\@savefreelist}
\let\real@dbflt=\@dbflt \let\end@dblfloat=\end@float
\let\@largefloatcheck=\relax
\let\if@boxedmulticols=\iftrue
\def\@dbflt{\let\@savefreelist\@freelist\real@dbflt}
\def\adjustnormalsize{\def\normalsize{\mathsurround=0pt \realnormalsize
 \parindent=0pt\abovedisplayskip=0pt\belowdisplayskip=0pt}%
 \def\phantompar{\csname par\endcsname}\normalsize}%
\def\lthtmltypeout#1{{\let\protect\string \immediate\write\lthtmlwrite{#1}}}%
\newcommand\lthtmlhboxmathA{\adjustnormalsize\setbox\sizebox=\hbox\bgroup\kern.05em }%
\newcommand\lthtmlhboxmathB{\adjustnormalsize\setbox\sizebox=\hbox to\hsize\bgroup\hfill }%
\newcommand\lthtmlvboxmathA{\adjustnormalsize\setbox\sizebox=\vbox\bgroup %
 \let\ifinner=\iffalse \let\)\liih@math }%
\newcommand\lthtmlboxmathZ{\@next\next\@currlist{}{\def\next{\voidb@x}}%
 \expandafter\box\next\egroup}%
\newcommand\lthtmlmathtype[1]{\gdef\lthtmlmathenv{#1}}%
\newcommand\lthtmllogmath{\dimen0\ht\sizebox \advance\dimen0\dp\sizebox
  \ifdim\dimen0>.95\vsize
   \lthtmltypeout{%
*** image for \lthtmlmathenv\space is too tall at \the\dimen0, reducing to .95 vsize ***}%
   \ht\sizebox.95\vsize \dp\sizebox\z@ \fi
  \lthtmltypeout{l2hSize %
:\lthtmlmathenv:\the\ht\sizebox::\the\dp\sizebox::\the\wd\sizebox.\preveqno}}%
\newcommand\lthtmlfigureA[1]{\let\@savefreelist\@freelist
       \lthtmlmathtype{#1}\lthtmlvboxmathA}%
\newcommand\lthtmlpictureA{\bgroup\catcode`\_=8 \lthtmlpictureB}%
\newcommand\lthtmlpictureB[1]{\lthtmlmathtype{#1}\egroup
       \let\@savefreelist\@freelist \lthtmlhboxmathB}%
\newcommand\lthtmlpictureZ[1]{\hfill\lthtmlfigureZ}%
\newcommand\lthtmlfigureZ{\lthtmlboxmathZ\lthtmllogmath\copy\sizebox
       \global\let\@freelist\@savefreelist}%
\newcommand\lthtmldisplayA{\bgroup\catcode`\_=8 \lthtmldisplayAi}%
\newcommand\lthtmldisplayAi[1]{\lthtmlmathtype{#1}\egroup\lthtmlvboxmathA}%
\newcommand\lthtmldisplayB[1]{\edef\preveqno{(\theequation)}%
  \lthtmldisplayA{#1}\let\@eqnnum\relax}%
\newcommand\lthtmldisplayZ{\lthtmlboxmathZ\lthtmllogmath\lthtmlsetmath}%
\newcommand\lthtmlinlinemathA{\bgroup\catcode`\_=8 \lthtmlinlinemathB}
\newcommand\lthtmlinlinemathB[1]{\lthtmlmathtype{#1}\egroup\lthtmlhboxmathA
  \vrule height1.5ex width0pt }%
\newcommand\lthtmlinlineA{\bgroup\catcode`\_=8 \lthtmlinlineB}%
\newcommand\lthtmlinlineB[1]{\lthtmlmathtype{#1}\egroup\lthtmlhboxmathA}%
\newcommand\lthtmlinlineZ{\egroup\expandafter\ifdim\dp\sizebox>0pt %
  \expandafter\centerinlinemath\fi\lthtmllogmath\lthtmlsetinline}
\newcommand\lthtmlinlinemathZ{\egroup\expandafter\ifdim\dp\sizebox>0pt %
  \expandafter\centerinlinemath\fi\lthtmllogmath\lthtmlsetmath}
\newcommand\lthtmlindisplaymathZ{\egroup %
  \centerinlinemath\lthtmllogmath\lthtmlsetmath}
\def\lthtmlsetinline{\hbox{\vrule width.1em \vtop{\vbox{%
  \kern.1em\copy\sizebox}\ifdim\dp\sizebox>0pt\kern.1em\else\kern.3pt\fi
  \ifdim\hsize>\wd\sizebox \hrule depth1pt\fi}}}
\def\lthtmlsetmath{\hbox{\vrule width.1em\kern-.05em\vtop{\vbox{%
  \kern.1em\kern0.8 pt\hbox{\hglue.17em\copy\sizebox\hglue0.8 pt}}\kern.3pt%
  \ifdim\dp\sizebox>0pt\kern.1em\fi \kern0.8 pt%
  \ifdim\hsize>\wd\sizebox \hrule depth1pt\fi}}}
\def\centerinlinemath{%
  \dimen1=\ifdim\ht\sizebox<\dp\sizebox \dp\sizebox\else\ht\sizebox\fi
  \advance\dimen1by.5pt \vrule width0pt height\dimen1 depth\dimen1 
 \dp\sizebox=\dimen1\ht\sizebox=\dimen1\relax}

\def\lthtmlcheckvsize{\ifdim\ht\sizebox<\vsize 
  \ifdim\wd\sizebox<\hsize\expandafter\hfill\fi \expandafter\vfill
  \else\expandafter\vss\fi}%
\providecommand{\selectlanguage}[1]{}%
\makeatletter \tracingstats = 1 
\providecommand{\Tau}{\textrm{T}}
\providecommand{\Rho}{\textrm{R}}
\providecommand{\Chi}{\textrm{X}}
\providecommand{\Iota}{\textrm{J}}
\providecommand{\Beta}{\textrm{B}}
\providecommand{\Mu}{\textrm{M}}
\providecommand{\Eta}{\textrm{H}}
\providecommand{\Kappa}{\textrm{K}}
\providecommand{\Epsilon}{\textrm{E}}
\providecommand{\Alpha}{\textrm{A}}
\providecommand{\omicron}{\textrm{o}}
\providecommand{\Zeta}{\textrm{Z}}
\providecommand{\Omicron}{\textrm{O}}
\providecommand{\Nu}{\textrm{N}}


\begin{document}
\pagestyle{empty}\thispagestyle{empty}\lthtmltypeout{}%
\lthtmltypeout{latex2htmlLength hsize=\the\hsize}\lthtmltypeout{}%
\lthtmltypeout{latex2htmlLength vsize=\the\vsize}\lthtmltypeout{}%
\lthtmltypeout{latex2htmlLength hoffset=\the\hoffset}\lthtmltypeout{}%
\lthtmltypeout{latex2htmlLength voffset=\the\voffset}\lthtmltypeout{}%
\lthtmltypeout{latex2htmlLength topmargin=\the\topmargin}\lthtmltypeout{}%
\lthtmltypeout{latex2htmlLength topskip=\the\topskip}\lthtmltypeout{}%
\lthtmltypeout{latex2htmlLength headheight=\the\headheight}\lthtmltypeout{}%
\lthtmltypeout{latex2htmlLength headsep=\the\headsep}\lthtmltypeout{}%
\lthtmltypeout{latex2htmlLength parskip=\the\parskip}\lthtmltypeout{}%
\lthtmltypeout{latex2htmlLength oddsidemargin=\the\oddsidemargin}\lthtmltypeout{}%
\makeatletter
\if@twoside\lthtmltypeout{latex2htmlLength evensidemargin=\the\evensidemargin}%
\else\lthtmltypeout{latex2htmlLength evensidemargin=\the\oddsidemargin}\fi%
\lthtmltypeout{}%
\makeatother
\setcounter{page}{1}
\onecolumn

% !!! IMAGES START HERE !!!

\setcounter{chapter}{-2}
\stepcounter{chapter}
\stepcounter{section}
\stepcounter{chapter}
\stepcounter{section}
\stepcounter{section}
\stepcounter{section}
\stepcounter{section}
\stepcounter{section}
\stepcounter{section}
\stepcounter{section}
\stepcounter{section}
\stepcounter{chapter}
\stepcounter{section}
\stepcounter{section}
\stepcounter{subsection}
\stepcounter{section}
\stepcounter{subsection}
\stepcounter{section}
\stepcounter{section}
\stepcounter{subsection}
\stepcounter{section}
\stepcounter{section}
\stepcounter{subsection}
\stepcounter{section}
\stepcounter{section}
\stepcounter{section}
\stepcounter{section}
\stepcounter{chapter}
\stepcounter{section}
\stepcounter{section}
\stepcounter{section}
\stepcounter{subsection}
\stepcounter{subsection}
\stepcounter{subsection}
\stepcounter{subsection}
\stepcounter{subsection}
\stepcounter{section}
\stepcounter{subsection}
\stepcounter{subsection}
\stepcounter{subsection}
\stepcounter{subsection}
\stepcounter{subsection}
\stepcounter{subsection}
\stepcounter{subsection}
\stepcounter{section}
\stepcounter{subsection}
\stepcounter{subsection}
\stepcounter{section}
\stepcounter{subsection}
\stepcounter{subsection}
\stepcounter{subsection}
\stepcounter{subsection}
\stepcounter{subsection}
\stepcounter{section}
\stepcounter{subsection}
\stepcounter{subsection}
\stepcounter{subsection}
\stepcounter{subsection}
\stepcounter{section}
\stepcounter{subsection}
\stepcounter{section}
\stepcounter{chapter}
\stepcounter{section}
\stepcounter{section}
\stepcounter{subsection}
\stepcounter{subsection}
\stepcounter{subsection}
\stepcounter{subsection}
\stepcounter{subsection}
\stepcounter{section}
\stepcounter{section}
\stepcounter{subsection}
\stepcounter{section}
\stepcounter{section}
\stepcounter{chapter}
\stepcounter{section}
{\newpage\clearpage
\lthtmlinlinemathA{tex2html_wrap_inline32305}%

% latex2html id marker 32305
$ \reflectbox{R}$%
\lthtmlinlinemathZ
\lthtmlcheckvsize\clearpage}

\stepcounter{section}
\stepcounter{section}
\stepcounter{section}
\stepcounter{subsection}
\stepcounter{subsection}
\stepcounter{section}
\stepcounter{subsection}
\stepcounter{section}
\stepcounter{section}
\stepcounter{section}
\stepcounter{chapter}
\stepcounter{section}
\stepcounter{section}
\stepcounter{subsection}
\stepcounter{section}
\stepcounter{section}
\stepcounter{section}
\stepcounter{section}
\stepcounter{section}
\stepcounter{chapter}
\stepcounter{section}
\stepcounter{section}
\stepcounter{section}
\stepcounter{section}
\stepcounter{chapter}
\stepcounter{section}
\stepcounter{section}
\stepcounter{section}
\stepcounter{section}
\stepcounter{subsection}
\stepcounter{subsection}
\stepcounter{section}
\stepcounter{section}
\stepcounter{section}
\stepcounter{section}
\stepcounter{section}
\stepcounter{chapter}
\stepcounter{section}
\stepcounter{subsection}
\stepcounter{section}
\stepcounter{section}
\stepcounter{section}
\stepcounter{section}
\stepcounter{section}
\stepcounter{section}
\stepcounter{section}
\stepcounter{section}
\stepcounter{section}
\stepcounter{section}
\stepcounter{section}
\stepcounter{chapter}
\stepcounter{section}
\stepcounter{section}
\stepcounter{subsection}
\stepcounter{subsection}
\stepcounter{section}
\stepcounter{subsection}
\stepcounter{subsection}
\stepcounter{subsection}
\stepcounter{subsection}
\stepcounter{subsection}
\stepcounter{section}
\stepcounter{section}
\stepcounter{section}
\stepcounter{subsection}
\stepcounter{subsection}
\stepcounter{subsection}
\stepcounter{subsection}
\stepcounter{subsection}
\stepcounter{subsection}
\stepcounter{subsection}
\stepcounter{subsection}
\stepcounter{subsection}
\stepcounter{section}
\stepcounter{chapter}
\stepcounter{section}
\stepcounter{section}
\stepcounter{section}
\stepcounter{section}
\stepcounter{section}
\stepcounter{subsection}
\stepcounter{section}
\stepcounter{section}
\stepcounter{section}
\stepcounter{section}
\stepcounter{subsection}
\stepcounter{section}
\stepcounter{section}
\stepcounter{section}
\appendix
\stepcounter{chapter}
\stepcounter{section}
\stepcounter{section}
\stepcounter{section}
\stepcounter{section}
\stepcounter{section}
\stepcounter{section}
\stepcounter{section}
\stepcounter{section}
\stepcounter{section}
\stepcounter{subsection}
\stepcounter{subsection}
\stepcounter{subsection}
\stepcounter{subsection}
\stepcounter{subsection}
\stepcounter{section}
\stepcounter{section}
\stepcounter{section}
\stepcounter{section}
\stepcounter{section}
\stepcounter{section}
\stepcounter{section}
\stepcounter{section}
\stepcounter{section}
\stepcounter{section}
\stepcounter{section}
\stepcounter{section}
\stepcounter{section}
\stepcounter{section}
\stepcounter{section}
\stepcounter{section}
\stepcounter{section}
\stepcounter{section}
\stepcounter{section}
\stepcounter{section}
\stepcounter{section}
\stepcounter{section}
\stepcounter{section}
\stepcounter{section}
\stepcounter{section}
\stepcounter{section}
\stepcounter{section}
\stepcounter{section}
\stepcounter{section}
\stepcounter{section}
\stepcounter{section}
\stepcounter{section}
\stepcounter{section}
\stepcounter{section}
\stepcounter{subsection}
\stepcounter{subsection}
\stepcounter{subsection}
\stepcounter{subsection}
\stepcounter{chapter}
\stepcounter{section}
\stepcounter{section}
\stepcounter{section}
\stepcounter{section}
\stepcounter{section}
\stepcounter{section}
\stepcounter{section}
\stepcounter{section}
\stepcounter{section}
\stepcounter{section}
\stepcounter{section}
\stepcounter{section}
\stepcounter{section}
\stepcounter{chapter}
\stepcounter{section}

\end{document}
