
% apendiceA.tex
% This work is licensed under the Creative Commons Attribution-Noncommercial-Share Alike 3.0 License.
% To view a copy of this license, visit http://creativecommons.org/licenses/by-nc-sa/3.0/nz
% or send a letter to Creative Commons, 171 Second Street, Suite 300, San Francisco, California, 94105, USA.

\chapter{Migrando código a Python 3 con \codigo{2to3}} \label{ch:migcod}

\noindent Nivel de dificultad:\diflllll

\begin{citaCap}
    ``La vida es agradable. La muerte es tranquilidad. Es la transición lo que es problemático.'' \\
        ---\emph{atribuido a Isaac Asimov}
\end{citaCap}

\section{Inmersión}

 Han cambiado tantas cosas entre Python 2 y Python 3 que son pocos los programas que funcionan con ambos sin modificaciones. ¡Pero no te desalientes! Para ayudar con la transición, Python 3 incorpora una herramienta denomidada \codigo{2to3}, que analiza el código fuente en Python 2 y lo convierte en Python 3 tanto como puede. El capítulo \ref{ch:chardet}, se describe como ejecutar \codigo{2to3}, y muestra algunas cosas que este no puede resolver de forma automática. Este apéndice documenta aquello que sí puede convertir de forma automática.

\section{La sentencia \codigo{print}}

En Python 2, \codigo{print} era una sentencia. Lo que quisieras imprimir iba detrás de ella. En Python 3, \codigo{print()} es una función. Lo quieras imprimir debe pasarse como parámetro, igual que en cualquier otra función.

\begin{table}
  \centering
  \begin{tabular}{ c l l }
  \hline
  Notas & Python 2 & Python 3 \\
  \hline
  1 & \codigo{print}                & \codigo{print()} \\
  2 & \codigo{print 1}              & \codigo{print(1)} \\
  3 & \codigo{print 1, 2}           & \codigo{print(1, 2)} \\
  4 & \codigo{print >{}>sys.stderr, 1, 2, 3} & \codigo{print(1, 2, 3, file=sys.stderr)} \\
  \hline
  \end{tabular}
\end{table}
\FloatBarrier

\begin{enumerate}
  \item \emph{Línea 1:} imprime una línea vacía.
  \item \emph{Línea 2:} imprime un único valor.
  \item \emph{Línea 3:} imprime dos valores separados por espacios.
  \item \emph{Línea 4:} este es un poco complejo. En Python 2, si finalizabas la sentencia \codigo{print} con una coma se imprimían los valores separados por espacios, luego añadía un espacio al final, y paraba sin imprimir un retorno de carro\footnote{Técnicamente, es un poco más complicado. La sentencia \codigo{print} en Python 2 utilizaba un atributo que está ``deprecado'' denominado \codigo{softspace}. En lugar de imprimir un espacio, Python 2 activaba \codigo{sys.stdout.softspace} a 1. El carácter de espacio, no se imprimía hasta que tu aplicación no imprimiera algo más en la misma línea. Si la siguiente sentencia \codigo{print} imprimía un retorno de carro, \codigo{sys.stdout.softspace} pasaría a valer 0 y el espacio nunca se imprimiría. Probablemente nunca te dieras cuenta de la diferencia a menos que tu aplicación fuese sensible a la presencia o ausencia de espacios en blanco al final de las líneas en la salida generada por \codigo{print}.}. En Python 3, la forma de hacer esto es pasar \codigo{end=' '} como un parámetro de la función \codigo{print()}. El parámetro \codigo{end} tiene como valor por defecto \codigo{'\textbackslash n'} (el retorno de carro), por lo que su sustitución elimina el retorno de carro después de imprimir los demás parámetros.

  \item \emph{Línea 5:} en Python 2, puedes redirigir la salida a un flujo diferente de la salida estándar ---como \codigo{sys.stderr}--- mediante el uso de la sintaxis \codigo{>{}>nombre\_de\_flujo}. En Python 3, la forma de hacer esto es pasar el flujo como parámetro \codigo{file}. El valor por defecto de este parámetro es \codigo{sys.stdout} (la salida estándar), por lo que sustituirlo enviará la salida a otro flujo diferente.

\end{enumerate}


\section{Cadenas de caracteres Unicode}

Python 2 tiene dos tipos de cadenas de caracteres: Unicode y no Unicode. Python 3 solo tiene un tipo: Unicode.


\begin{table}
  \centering
  \begin{tabular}{ c l l }
  \hline
  Notas & Python 2 & Python 3 \\
  \hline
  1 & \codigo{u'PapayaWhip'}              & \codigo{'PapayaWhip'} \\
  2 & \codigo{ur'PapayaWhip\textbackslash foo'}             & \codigo{r'PapayaWhip\textbackslash foo'} \\
  \hline
  \end{tabular}
\end{table}
\FloatBarrier

\begin{enumerate}
  \item \emph{Línea 1:} las cadenas de caracteres Unicode se escriben directamente como cadenas de caracteres, que en Python 3 siempre son Unicode.
  \item \emph{Línea 2:} las cadenas de caracteres Unicode Raw (que en Python permiten evitar los caracteres de escape de la barra inclinada invertida) se convierten en cadenas de caracteres Raw, que en Python 3, siempre son Unicode.
\end{enumerate}
