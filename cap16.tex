% ch16.tex
% This work is licensed under the Creative Commons Attribution-Noncommercial-Share Alike 3.0 License.
% To view a copy of this license, visit http://creativecommons.org/licenses/by-nc-sa/3.0/nz
% or send a letter to Creative Commons, 171 Second Street, Suite 300, San Francisco, California, 94105, USA.

\chapter{Empaquetando librerías en Python}\label{ch:emplib}

\noindent Nivel de dificultad:\difllll

\begin{citaCap}
    ``Descubrirás que la vergüenza es como el dolor; solo lo sientes una vez.'' \\
        ---\emph{Marquesa de Merteuil, Las amistades peligrosas\footnote{\href{http://www.imdb.com/title/tt0094947/quotes}{http://www.imdb.com/title/tt0094947/quotes}}}
\end{citaCap}

\section{Inmersión}

Los verdaderos artistas lanzan productos. O eso dice Steve Jobs. ¿Quieres sacar una versión de un script, librería, framework o aplicación? Excelente. El mundo necesita más código en Python. Python 3 viene con un framework de empaquetado denominado \codigo{Distutils}, que sirve para muchas cosas:

\begin{itemize}
  \item Una herramienta de construcción (para ti).
  \item Una herramienta de instalación (para tus usuarios).
  \item Un formto de metadatos para los paquetes (para las herramientas de búsqueda).
  \item Y mas\ldots
\end{itemize}

Se integra con el Índice de Paquetes de Python ``(\codigo{PyPI})''\footnote{\href{http://pypi.python.org/}{http://pypi.python.org/}}, un repositorio central para las librerías de Python de código abierto.

Todos estos aspectos de \codigo{Distutils} giran en torno al \emph{script de setup}, que tradicionalmente se ha denominado \codigo{setup.py}. De hecho, ya has visto varios scripts de setup (configuración) a lo largo de este libro. Utilizaste \codigo{Distutils} para instalar \codigo{httplib2} en el apartado \ref{sec:httplib2}, Introducción a \codigo{httplib2}. También lo utilizaste para instalar \codigo{chardet} en el capítulo \ref{ch:chardet}, Caso de estudio: migrar chardet a Python 3.

En este capítulo aprenderás cómo funcionan los scripts de configuración para las librerías \codigo{chardet} y \codigo{httplib2}, y recorrerás los pasos necesarios para generar una versión de tu propio software en Python.

\begin{lstlisting}[language=Python,breaklines=true]
# chardet's setup.py
from distutils.core import setup
setup(
    name = "chardet",
    packages = ["chardet"],
    version = "1.0.2",
    description = "Universal encoding detector",
    author = "Mark Pilgrim",
    author_email = "mark@diveintomark.org",
    url = "http://chardet.feedparser.org/",
    download_url = "http://chardet.feedparser.org/download/python3-chardet-1.0.1.tgz",
    keywords = ["encoding", "i18n", "xml"],
    classifiers = [
        "Programming Language :: Python",
        "Programming Language :: Python :: 3",
        "Development Status :: 4 - Beta",
        "Environment :: Other Environment",
        "Intended Audience :: Developers",
        "License :: OSI Approved :: GNU Library or Lesser General Public License (LGPL)",
        "Operating System :: OS Independent",
        "Topic :: Software Development :: Libraries :: Python Modules",
        "Topic :: Text Processing :: Linguistic",
        ],
    long_description = """\
Universal character encoding detector
-------------------------------------

Detects
 - ASCII, UTF-8, UTF-16 (2 variants), UTF-32 (4 variants)
 - Big5, GB2312, EUC-TW, HZ-GB-2312, ISO-2022-CN (Traditional and Simplified Chinese)
 - EUC-JP, SHIFT_JIS, ISO-2022-JP (Japanese)
 - EUC-KR, ISO-2022-KR (Korean)
 - KOI8-R, MacCyrillic, IBM855, IBM866, ISO-8859-5, windows-1251 (Cyrillic)
 - ISO-8859-2, windows-1250 (Hungarian)
 - ISO-8859-5, windows-1251 (Bulgarian)
 - windows-1252 (English)
 - ISO-8859-7, windows-1253 (Greek)
 - ISO-8859-8, windows-1255 (Visual and Logical Hebrew)
 - TIS-620 (Thai)

This version requires Python 3 or later; a Python 2 version is available separately.
"""
)
\end{lstlisting}

\cajaTextoAncho{\codigo{chardet} y \codigo{httplib2} son librerías de código abierto, pero no existe ningún requisito para que tú liberes tus propias librerias Python bajo la licencia que quieras. El proceso descrito en este capítulo funcionará para cualquier software Python, independientemente del tipo de licencia.}

\section{Cosas que \codigo{Distutils} no puede hacer por ti}

Distribuir tu primer paquete Python es un proceso intimidante (la segunda es un poco más sencillo). \codigo{Distutils} intenta automatizarlo al máximo, pero hay algunas cosas que tienes que hacer por ti mismo.

\begin{itemize}
  \item \textbf{Elige la licencia}. Este es un tema complejo, cargado de política y peligro. Si deseas liberar tu software como código abierto, te ofrezco humildemente cinco consejos:

    \begin{enumerate}
      \item No escribas tu propia licencia.
      \item No escribas tu propia licencia.
      \item No escribas tu propia licencia.
      \item No tiene que ser \codigo{GPL} pero tiene que ser compatible\footnote{\href{http://www.dwheeler.com/essays/gpl-compatible.html}{http://www.dwheeler.com/essays/gpl-compatible.html}}.
      \item No escribas tu propia licencia.
    \end{enumerate}

  \item \textbf{Clasifica tu software} utilizando el sistema de clasificación de PyPI. Explicaré lo que significa más adelante en este capítulo.
  \item \textbf{Escribe un fichero ``readme'' (léeme)}. No te saltes esto. Como mínimo, debería dar a tus usuarios una idea de lo que hace tu software y cómo instalarlo.
\end{itemize}

\section{La estructura de directorios}

Para comenzar a empaquetar un software Python necesitas colocar tus ficheros y directores en el orden adecuado. La estructura de directorios de \codigo{httplib2} es la siguiente:


\begin{lstlisting}[breaklines=true]
httplib2/                 
|
+--README.txt            
|
+--setup.py             
|
+--httplib2/           
   |
   +--__init__.py
   |
   +--iri2uri.py
\end{lstlisting}

\begin{enumerate}
  \item \emph{Línea 1:} Crea un directorio raíz para contenerlo todo. Dale el mismo nombre que el de tu módulo Python.
  \item \emph{Línea 3:} Para acomodar a tus usuarios de windows, el fichero ``README'' debería incluir la extensión \codigo{.txt}, y debería utilizar el estilo de retorno de caracteres de Windows. Solo porque utilices un moderno editor de texto que se ejecuta desde la línea de comando e incluya su propio lenguaje de macros, eso no significa que tengas que hacerle la vida más difícil a tus usuarios (Tus usuarios usan el block de notas de Windows, triste pero cierto). Incluso aunque estés en Linux o en Mac OS X, tu editor de textos seguro que tiene una opción para grabar los ficheros con retornos de carro al estilo de Windows.

  \item \emph{Línea 5:} Tu script de configuración de \codigo{Distutils} debería llamarse \codigo{setup.py}, a menos que tengas una buena razón para no hacerlo así. Pero no la tienes.
  \item \emph{Línea 7:} Si tu paquete Python es un único fichero \codigo{.py}, deberías ponerlo en el directorio ``raíz'' junto al fichero ``README'' y tu script de configuración. Pero la librería \codigo{httplib2} no es un único fichero; es un módulo multifichero. Simplemente añade el directorio \codigo{httplib2} dentro del directorio raíz. De este modo tendrás el fichero \codigo{\_\_init\_\_.py} dentro de un directorio \codigo{httplib2} que, a su vez, está dentro del directorio raíz \codigo{httplib2}. Esto no es un problema; de hecho, simplifica el proceso de empaquetado.
\end{enumerate}

La estructura de directorios de \codigo{chardet} es un poco diferente. Como \codigo{httplib2} es un módulo multifichero, así que hay un directorio \codigo{chardet} dentro del directorio ``raíz'' \codigo{chardet}. Además del fichero \codigo{README.txt}, \codigo{chardet} dispone de una documentación formateada en HTML, en el directorio \codigo{docs}. Este contiene varios ficheros \codigo{.html} y \codigo{css} y un subdirectorio \codigo{images} que contiene varios ficheros \codigo{.png} y \codigo{.gif} (esto será importante después). También, manteniendo la convención del software liberado bajo licencia \codigo{gpl}, tiene un fichero separado denominado \codigo{COPYING.txt} que contiene el texto completo de la licencia \codigo{GPL}.


\begin{lstlisting}[breaklines=true]
chardet/
|
+--COPYING.txt
|
+--setup.py
|
+--README.txt
|
+--docs/
|  |
|  +--index.html
|  |
|  +--usage.html
|  |
|  +--images/ ...
|
+--chardet/
   |
   +--__init__.py
   |
   +--big5freq.py
   |
   +--...
\end{lstlisting}

\section{Escribiendo el script de configuración}

El script de configuración de \codigo{Distutils} es un script en Python. En teoría puede hacer cualquier cosa que se pueda hacer en Python. En la práctica, debería hacer lo mínimo posible. Debería ser un script ``aburrido''. Cuanto más exótico sea tu proceso de instalación, tanto más complejo serán los mensajes de error durante el mismo.

La primera línea de cualquier fichero de script de configuración es:


\begin{lstlisting}[language=Python,breaklines=true]
form distutils.core import setup
\end{lstlisting}

Esta línea importa la función \codigo{setup()}, que es el punto de entrada principal a \codigo{Distutils}. El 95\% de los ficheros de configuración de \codigo{Distutils} constan simplemente de una llamada al método \codigo{steup()} y nada más (me he inventado la estadística, pero si tu script hace más cosas, deberías tener una buena razón. ¿La tienes? Seguramente no).

La función \codigo{setup()} puede tomar docenas de parámetros\footnote{\href{http://docs.python.org/3.1/distutils/apiref.html\#distutils.core.setup}{http://docs.python.org/3.1/distutils/apiref.html\#distutils.core.setup}}. Para mantener la cordura de los que tengan que leerla, deberías usar argumentos con nombre para cada uno que uses. Esto no es solamente una convención; es un requisito. Tu script de configuración fallará si intentas llamar a la función con argumentos sin nombre.

Se requieren, al menos, los siguientes:

\begin{itemize}
  \item \textbf{name}, es el nombre del paquete.
  \item \textbf{version}, el número de versión del paquete.
  \item \textbf{author\_email}, tu dirección de correo.
  \item \textbf{url}, la página web de tu proyecto. Esta puede ser la página del paquete en \codigo{PyPI} si no tienes una web separada para tu proyecto.
\end{itemize}

Aunque no se requieren, recomiendo que también incluyas los siguientes parámetros en tu script de configuración:

\begin{itemize}
  \item \textbf{description}, un resumen de una línea sobre el proyecto.
  \item \textbf{long\_description}, una cadena de caracteres de varias líneas en formato \codigo{reStructuredFormat}\footnote{\href{http://docutils.sourceforge.net/rst.html}{http://docutils.sourceforge.net/rst.html}}. \codigo{PyPI} convierte este formato a \codigo{HTML} y lo muestra en la página de tu paquete.
  \item \textbf{classifiers}, una lista de cadenas de caracteres formateadas de manera especial como se describe en la sección siguiente.
\end{itemize}

\cajaTextoAncho{Los metadatos del script de configuración están definidos en la PEP 314 \href{http://www.python.org/dev/peps/pep-0314/}{http://www.python.org/dev/peps/pep-0314/}}

Ahora veamos el script de configuración de \codigo{chardet}. Contiene todos los parámetros requeridos y los recomendados, más uno que no he mencionado aún: \codigo{packages}.


\begin{lstlisting}[language=Python,breaklines=true]
from distutils.core import setup
setup(
    name = 'chardet',
    packages = ['chardet'],
    version = '1.0.2',
    description = 'Universal encoding detector',
    author='Mark Pilgrim',
    ...
)
\end{lstlisting}

El parámetro \codigo{packages} destaca un solapamiento desafortunado en el vocabulario del proceso de distribución. Hemos estado hablando del ``paquete'' como la cosa que estamos construyendo (y potencialmente, listándola en el Índice de Paquetes de Python). Pero eso no es lo que el parámetro \codigo{packages} quiere expresar. Se refiere al hecho de que el módulo \codigo{chardet} es un módulo multifichero, algunas veces conocido como\ldots un ``paquete''. El parámetro \codigo{packages} le indica a \codigo{Distutils} que incluya el directorio \codigo{chardet}, su fichero \codigo{\_\_init\_\_.py} y todos los ficheros \codigo{.py} que constituyen el módulo \codigo{chardet}. Esto es muy importante; toda esta parrafada sobre la documentación y los metadatos es irrelevante si te olvidas de incluir el código propiamente dicho.
